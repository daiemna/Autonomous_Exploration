\section{Problem Description}

\subsection{Problem Addressed}
Our objective is to implement exploration scheme for ground mobile robot, which enables it to
autonomously explore the given indoor environment using its perception sensors.
Using these laser scans the robot should be able to build a 2D map of the surroundings.
The end result of this endeavor will be a module which is able to devise an exploration plan.
The efficiency of the exploration strategy is also a concern which can be measured using metrics
proposed in \cite{Yan2015}.
\par
Formally \textit{Occupancy Grid Maps} represent the occupation probability of each zone of the
environment within a grid \cite{Juliae2012}. One of the reasons of choosing a strategy is that
it can work with occupancy grids. The mobile robots for which the autonomous exploration is needed
are care-o-bot and you-bot which are actively used by RoboCup @Home and @Work team in
Bonn-Rhein-Sieg University of Applied Science.

\subsection{Approach}
Following is just an abstract, please comment:
\begin{itemize}
	\item choose metrics of comparison for @home or @work.
	\item setup a simulator for testing.
	\item choose and implement a nearest frontier base explorer.
	\item choose and implement a cost-utility based explorer, which is an extension of frontier based
	algorithm.
	\item choose and implement a RL based explorer, would need a lot of time for parameter tuning here.
	\item compare by decided metrics.
\end{itemize}

% \subsection{Use Cases}

\subsection{Expected Results}
\begin{itemize}
	\item Minimum:
	\begin{itemize}
		\item Autonomous Exploration with frontier cost-utility.
	\end{itemize}
	\item Expected:
	In addition to the above:
	\begin{itemize}
		\item Autonomous Exploration with Reinforcement Learning.
	\end{itemize}
	\item Maximum:
	In addition to the above:
	\begin{itemize}
		\item Comparison of both RL vs. cost-utility 
	\end{itemize}
\end{itemize}
